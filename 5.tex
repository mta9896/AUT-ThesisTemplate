 
\cchapter{نتیجه‌گیری و کارهای آینده}
\label{chap:conclusion}
\pagebreak
در این کار، یک سیستم تشخیص و دسته‌بندی آریتمی قلبی پیاده‌سازی شد. این سیستم توانایی کارکردن به صورت بی‌درنگ را دارد و در طول مدت‌زمان کم‌تر از ۲ ثانیه، کلاس آریتمی ضربان دریافت‌شده را تشخیص می‌دهد. دقت الگوریتم دسته‌بندی پیاده‌سازی‌شده به طور میانگین ۰/۴۴ به دست آمد، هم‌چنین حساسیت میانگین این دسته‌بندی ۰/۷۳ و  \lr{j$\kappa$ index} آن ۰/۴۳ محاسبه شد. 

این سیستم به دو بخش کلی پیش‌پردازش بر روی سخت‌افزار و پردازش اصلی بر روی سرور تقسیم می‌شود. بخش اول بر روی یک ماژول \lr{ESP8266} پیاده شد که باید به یک سنسور دیجیتال ضربان قلب متصل شود. در این بخش یک الگوریتم تشخیص \lr{QRS} با روش پن-تامپکینز و با هدف تشخیص قله‌های \lr{R} هر ضربان بر روی سیگنال نوار قلب اجرا شد و نتیجه‌ی این پیش‌پردازش به سرور ارسال گردید.

در بخش دوم، یک مدل دسته‌بندی \lr{SVM} به صورت آفلاین آموزش داده شد و سپس مدل آموزش‌داده‌شده به صورت فایل ذخیره شده و به سرور انتقال داده شد. یک کد سمت سرور برای دریافت درخواست‌ها و پیش‌بینی کلاس ضربان‌‌های جدید پیاده‌سازی شد.

الگوریتم دسته‌بندی بر روی پایگاه‌داده‌ی \lr{MIT-BIH} اجرا شد و برای تقسیم داده‌های این پایگاه‌داده به دو مجموعه‌ی آموزش و تست، از الگوی بین‌بیماری بهره گرفته شد. برای ارزیابی کارایی این دسته‌بندی، از معیار  \lr{j$\kappa$ index} استفاده شد که نمای مناسبی از میزان کارایی یک الگوریتم دسته‌بندی آریتمی ارایه می‌کند.

از مهم‌ترین نیازمندی‌های این پروژه، سرعت بالا و بی‌درنگ بودن تشخیص آریتمی است که با توجه به زمان پاسخ به‌دست‌آمده، می‌توان گفت این نیازمندی برآورده شده‌است. هم‌چنین قابل‌حمل‌بودن سخت‌افزار همراه بیمار  از نیازمندی‌های دیگر بود که با پیاده‌سازی بخش سخت‌افزاری کار بر روی ماژول \lr{ESP8266} که حجم و مساحت کوچکی دارد، تا حد خوبی برآورده شده‌است.

نیازمندی دیگر این کار، دقت بالای الگوریتم دسته‌بندی است. در این کار در مرحله‌ی استخراج ویژگی، تنها از بازه‌های \lr{RR} ضربان‌های قلب استفاده شد و ویژگی‌هایی با استفاده از این بازه‌ها استخراج شدند. دلیل این امر، کارایی بالایی بود که این ویژگی‌ در کارهای گذشته از خود نشان داده‌است. حجم پایین ویژگی‌های استخراج شده و سبک‌تر بودن محاسبات مورد نیاز بر روی سخت‌افزار همراه بیمار  نیز باعث شد ویژگی \lr{RR} مناسب تشخیص داده شود، چرا که ما را به هدف بی‌درنگ‌بودن سیستم و قابل‌حمل‌بودن سخت‌افزار آن نزدیک می‌نماید.

در برخی از الگوریتم‌های دسته‌بندی موجود از ویژگی‌های دیگری چون ویژگی‌هایی که شکل موج سیگنال را مورد بررسی قرار می‌دهند، و یا ترکیب چندین ویژگی از این دست استفاده شده‌است. این ویژگی‌ها به دلیل توصیف بهتر سیگنال ضربان قلب (به نسبت ویژگی \lr{RR}) بعضا مدل‌هایی با  \lr{j$\kappa$ index} بالاتری از آن‌چه در این کار به دست آمد ارایه کرده‌اند، و در حال حاضر بالاترین  \lr{j$\kappa$ index} به‌دست‌آمده در کارهای گذشته ۰/۶۶۳ است.

در کارهای آینده می‌توان از چند مدل \lr{SVM} که هر یک با یک مجموعه از ویژگی‌ها آموزش داده شده اند و ترکیب نتایج آن‌ها، برای ساخت یک مدل \lr{SVM} قوی‌تر بهره برد. در کار پیش رو، در مرحله‌ی تشخیص \lr{QRS}، تنها از ولتاژ یک لید (\lr{MLII}) استفاده شد که واضح‌ترین ترکیب \lr{QRS} را بین لیدهای قلبی نمایش می‌دهد. در آینده می‌توان الگوریتم‌هایی طراحی نمود که از  لیدهای بیشتری بهره می‌برند، و تاثیر این کار را بر بالارفتن دقت تشخیص \lr{QRS} بررسی کرد.
