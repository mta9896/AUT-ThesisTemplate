
%برای گزارشات ماهانه mscR و برای نسخه نهایی msc انتخاب شود
\documentclass[oneside,openany,bsc]{AUT-Thesis}
\input{commands}

%\university{دانشگاه صنعتی امیرکبیر (پلی تکنیک تهران)}
\faculty{دانشکده‌ی مهندسی کامپیوتر و فناوری اطلاعات}

%\baselineskip=2cm
\title{
طراحی و پیاده‌سازی یک سیستم تشخیص بی‌درنگ آریتمی قلبی بر بستر اینترنت اشیا
}

%برای گزارشات ماهانه به این صورت نسخه مشخص میشود
%\reportVersion{چهارم}

\subject{مهندسی کامپیوتر}
\field{معماری کامپیوتر}


\name{مرضیه}
\surname{تاجیک}
\firstsupervisor{دکتر محمود ممتازپور}
\firstadvisor{دکتر مرتضی صاحب‌زمانی}
\thesisdate{اسفند ۱۳۹۷}

\fa-abstract{تشخیص آریتمی‌های قلبی در صورتی که به موقع انجام شود، می‌تواند از عواقب خطرناک بعدی این دسته از بیماری‌ها جلوگیری نماید. هدف  از این پروژه، طراحی یک سیستم تشخیص آریتمی است که بتواند به صورت بی‌درنگ ضربان‌‌ها را دریافت کرده و آن‌ها را دسته‌بندی کند‌. در این پروژه بستری برای تشخیص بی‌درنگ آریتمی قلبی طراحی و پیاده‌سازی شده است. این سیستم با استفاده از اینترنت اشیا و در دو بخش طراحی می‌شود. بخش اول پیش‌پردازش‌هایی را بر روی سیگنال ضربان قلب دریافت‌شده از بیمار انجام می‌دهد و  پیاده‌سازی آن بر روی یک میکروکنترلر است. در بخش دوم، پردازش‌های اصلی شامل استخراج ویژگی‌های سیگنال پیش‌پردازش‌شده و اجرای یک الگوریتم دسته‌بندی بر روی آن‌ها، بر روی یک سرور پیاده‌سازی می‌شود. این دو بخش به کمک اینترنت با یکدیگر ارتباط دارند. الگوریتم دسته‌بندی استفاده‌شده می‌تواند با دقت میانگین ۰/۴۴، حساسیت ۰/۷۳و اندیس   \lr{j$\kappa$}  برابر با ۰/۴۳، کلاس آریتمی ضربان‌های ارسال شده را تشخیص دهد‌. زمان پاسخ سیستم  از زمان دریافت نمونه‌ی سیگنال نوار قلب تا تشخیص آریتمی حدود ۲ ثانیه تخمین زده می‌شود، هم‌چنین  در این کار یک رابط کاربری  در قالب اپلیکیشن وب طراحی و ساخته شده است که نتایج تشخیص آریتمی آخرین ضربان‌های دریافت‌شده را به پزشک و بیمار نمایش می‌دهد.
}


\keywords{اینترنت اشیا، آریتمی قلبی، الگوریتم دسته‌بندی، تشخیص بی‌درنگ}

%%%% english
\en-abstract{}
\latinkeywords{}

%\latinuniversity{Amirkabir University of Technology (Tehran Polytechnic)}
\latinfaculty{Faculty of Robotics Engineering}
\latintitle{Real-time Detection and Localization of Traffic Panels and Persian Text in Street-Level Videos}
\latinsubject{Robotics Engineering}
\latinname{\lr{Navid}}
\latinsurname{\lr{Khazaee Korghond}}
\firstlatinsupervisor{Prof. Reza Safabakhsh}
\latinthesisdate{February 2016}

\begin{document}

\pagenumbering{harfi}
\firstPage
\abstractPage
%\davaranPage
\tableofcontents \newpage
\listoffigures \newpage
\listoftables \newpage
\pagenumbering{arabic}

%You can write your chapters here to see the output, then when finished, transfer the chapter to a seperate file and include it here. While writing new chapters, you can avoid including previous ones to speed up the compile time. 

%---------------------- chap 1 ----------------------
\cchapter{مقدمه}
\pagebreak

بر اساس آمارهای سازمان سلامت جهانی\LTRfootnote {World Health Organization} بیماری‌های قلبی-عروقی \LTRfootnote{Cardiovascular diseases} رتبه‌ی اول را در بین بیماری‌های کشنده در سطح جهان دارند. برای مثال در سال ۲۰۱۶ حدود ۱۷/۹ میلیون مرگ (حدود ۳۱٪ آمار کلی فوت) به علت بیماری‌های قلبی عروقی تخمین زده شده‌است. \cite{WHO} حدود ۲۵٪ این تعداد را مرگ‌های ناگهانی قلبی (SCD) تشکیل می‌دهند. \cite{Srinivasan2018} در چنین شرایطی، بیمار در طول مدت یک ساعت پس از آغاز علایم دچار ایست قلبی می‌شود. 
علت اصلی ایست‌های قلبی ناگهانی، آریتمی‌های قلبی هستند. \cite{Cleveland} این عبارت به دسته‌ای از بیماری‌های قلبی اطلاق می‌شود که در آن‌ها، اختلالاتی در آهنگ طبیعی تپش قلب به وجود می‌آید. با وجود این که بیشتر آریتمی‌ها بی خطر هستند، در برخی موارد در صورت عدم رسیدگی می‌توانند مرگبار باشند. به همین دلیل، تشخیص و درمان به موقع آن‌ها از اهمیت بالایی برخوردار است. 

یکی از رایج‌ترین و مهم‌ترین ابزارها در تشخیص بیماری‌های قلبی-عروقی، سیگنال نوار قلب \LTRfootnote{Electrocardiogram (ECG)} است. \cite {Elgendi2014} قلب ماهیچه‌ای است که با تحریک سیگنال‌های الکتریکی، به صورت منظم در حال تپش است. این فعالیت الکتریکی قلب، باعث ایجاد نوساناتی متناوب در پتانسیل الکتریکی سطح پوست می‌شود. این نوسانات  را می‌توان به کمک الکترودهایی که روی پوست قرارمی‌گیرند، اندازه‌گیری و در قالب سیگنال نوار قلب ثبت نمود. 

تحلیل سیگنال نوار قلب، اطلاعات مفیدی در راستای تشخیص آریتمی و نوع آن فراهم می‌کند. \cite{Mondejar} از همین روی، در چند دهه‌ی گذشته پژوهش‌های گسترده‌ای بر روی طراحی سیستم‌های خودکار تشخیص آریتمی صورت گرفته‌است. در این سیستم‌ها، ابتدا سیگنال نوار قلب به وسیله‌ی الکترودها و تجهیزات مخصوص، از بیمار گرفته شده و فیلترهایی به جهت حذف انواع نویزها بر روی آن اعمال می‌شود. قدم بعدی، استخراج تک‌تک ضربان‌های یک سیگنال نوار قلب است. در این مرحله یک الگوریتم قطعه‌بندی \LTRfootnote{Segmentation} بر روی نوار قلب اجرا می‌شود.

هر تک‌ضربان قلب شامل تعدادی موج است که در کنار هم نوسانات ضربان را تشکیل می‌دهند. موج‌های \lr{Q}، \lr{R} و \lr{S} مهم‌ترین موج‌ها در تحلیل نوار قلب هستند.\cite{Mondejar} به مجموعه‌ی این سه موج در کنار هم، ترکیب \lr{QRS} گفته می‌شود. به دلیل اهمیت این ترکیب در تشخیص انواع آریتمی، بخش مهمی از کارهای گذشته به تشخیص خودکار این ترکیب در ضربان قلب اختصاص داده شده‌است. معمولا در مرحله‌ی قطعه‌بندی موقعیت زمانی ترکیب \lr{QRS} هر ضربان و یا قله‌های \lr{R} در ضربان‌های متوالی تشخیص داده می‌شود.

 در مرحله‌ی بعد، مجموعه‌ای از ویژگی‌ها از هر یک از ضربان‌ها استخراج شده و به یک دسته‌بند \LTRfootnote{Classifier} داده‌می‌شود. این دسته‌بند نوع ضربان که خروجی نهایی این سیستم است را تعیین می‌کند. دیاگرام معماری کلی چنین سیستمی در شکل \ref{classifierPicture} قابل مشاهده است. 
 
\begin{figure}[!htb]
\centering
\includegraphics[width=15cm]{Figures/classifier.png}
\caption{مراحل اصلی یک سیستم خودکار تشخیص آریتمی\cite{Mondejar}}
\label{fig:classifierPicture}
\end{figure}

در این پروژه، هدف بر این است که بستری بی‌درنگ برای تشخیص آریتمی فراهم شود. به دلیل اهمیت تشخیص سریع در برخی از انواع خطرناک آریتمی، به خصوص آریتمی‌هایی که منجر به ایست ناگهانی قلبی می‌شوند، بی‌درنگ بودن این سیستم حایز اهمیت است. این امر نیازمند این است که تمامی بخش‌های سیستم، شامل بخش پیش‌پردازش،‌ بخش استخراج ویژگی و الگوریتم دسته‌بندی، همگی توانایی کارکردن به صورت برخط را داشته‌باشند. به بیان دیگر این سیستم به شکل خط‌لوله‌ای طراحی شده‌است که در آن ضربان‌ها به صورت پی‌در‌پی تولید، پردازش‌ و دسته‌بندی می‌شوند. پس از پیاده‌سازی، تاخیر هر یک از بخش‌ها اندازه‌گیری شده و تاخیر کلی سیستم تخمین زده می‌شود. 
همان‌طور که گفته‌شد، اولین بخش سیستم، بخش پیش‌پردازش ضربان قلب است. در این بخش یک الگوریتم تشخیص QRS طبق روش پن و تامپکینز (منبع)  پیاده‌سازی شده‌است. این الگوریتم یک روش بی‌درنگ است که سیگنال دیجیتال‌شده‌ی نوار قلب را به عنوان ورودی دریافت کرده و موقعیت زمانی قله‌های R را در هر یک از ضربان‌ها تشخیص می‌دهد. فاصله‌ی هر قله‌ی R تشخیص‌داده‌شده با قله‌ی بعدی و قبلی خود، که تحت عنوان فاصله‌ی R-R شناخته می‌شود، مهم‌ترین ویژگی در تشخیص نوع ضربان قلب (نوع آریتمی آن ضربان) است. (نیازمند منبع) 
در مرحله‌ی بعد، ویژگی‌های مورد نظر، از فواصل R-R تشخیص‌داده‌شده استخراج می‌گردند. این ویژگی‌ها سپس به یک دسته‌بند SVM که پیش‌تر مراحل یادگیری را طی کرده‌است، داده می‌شوند و دسته‌بند به کمک ویژگی‌های ورودی، نوع آریتمی را تشخیص می‌دهد. ضربان‌های دارای آریتمی انواع متعددی دارند که در ۵ دسته‌ی کلی دسته‌بندی می‌شوند. خروجی سیستم ما، تشخیص یکی از این دسته‌ها برای هر ضربان قلب است.

یکی از نیازمندی‌های بستر طراحی‌شده در این پروژه، این است که بتوان سیستمی قابل حمل و قابل استفاده‌ی آسان برای بیمار را بر روی این بستر پیاده‌سازی کرد. برای پیاده‌سازی این کاربرد، اینترنت اشیا راه‌حل مناسبی تشخیص داده‌شد. در چنین کاربردی، انتظار می‌رود بیمار دستگاهی ساده در اختیار داشته‌باشد که ضربان قلب او را دریافت کرده و پیش‌پردازش‌هایی ساده را بر روی آن پیاده نماید، و پردازش‌های پیچیده‌تر برای تشخیص آریتمی، بر عهده‌ی یک سرور با توان پردازشی بالاتری باشد. سپس نتایج این پردازش‌ها به اطلاع بیمار و پزشک او برسد.

 برای این منظور، معماری کلی سیستم به دو بخش تقسیم شد. بخش اول سیستم، وظیفه‌ی دریافت ضربان قلب از بیمار،‌ انجام پیش‌پردازش‌هایی بر روی آن، و در انتها ارسال نتایج پیش‌پردازش به سرور را دارد. این بخش به صورت سخت‌افزاری پیاده شده‌است و کافی است یک حس‌گر دیجیتال ضربان قلب، برای دریافت ضربان قلب بیمار به آن متصل شود. 
بخش دوم سیستم با استفاده از الگوریتم‌های یادگیری ماشین، و بر روی یک سرور پیاده‌سازی شده‌است. در این سرور، نتایج پیش‌پردازش‌های انجام شده در بخش قبل در سرور دریافت شده و ویژگی‌های هر ضربان استخراج می‌شود. سپس با استفاده از این ویژگی‌ها، عمل دسته‌بندی ضربان‌ها انجام می‌شود.

کارهای گذشته در قسمت فیچر:
در کارهای گذشته، ویژگی‌های مختلفی برای توصیف ضربان قلب معرفی شده‌اند. تبدیل موجک (منبع) و آمارهای مرتبه بالاتر (HOS) (منبع) از جمله‌ی این ویژگی‌ها هستند. در تبدیل موجک، اطلاعاتی هم در حوزه‌ی زمان و هم در حوزه‌ی فرکانس از سیگنال استخراج می‌شود. 
(توضیحات بیشتر در مورد اینترنت اشیا)
در برخی از پژوهش‌ها از بازه‌های R-R به عنوان ویژگی استفاده شده‌است (منبع) (منبع ۶۷ تا ۷۹ سوروی). آریتمی قلبی باعث برهم‌خوردن آهنگ تپش و در نتیجه‌ی آن، توازن منحنی ضربان قلب می‌شود، و ابن اتفاق تاثیر مستقیمی بر روی نوسانات فاصله‌های قله‌های R می‌گذارد. (منبع ۱ سوروی) به همین دلیل ویژگی R-R ظرفیت بالایی برای تشخیص انواع آریتمی دارد. این ویژگی در بین ویژگی‌های به کار گرفته‌شده پراستفاده‌ترین است. (منبع ۱۵ مرده خوار)
این ویژگی، نسبت به ویژگی‌های morphological حجم کم‌تری به خود اختصاص می‌دهد. در کار پیش رو، ویژگی‌های استخراج شده از ضربان قلب بیمار در مرحله‌ی پیش‌پردازش، به یک سرور فرستاده می‌شوند تا پردازش‌های بیش‌تر بر روی آن‌ها انجام شود. از این رو لازم است حجم داده‌های ارسال‌شده، و پیرو آن، حجم ویژگی‌های استخراج‌شده کنترل شود. در صورتی که ویژگی‌های استخراج‌شده حجم زیادی داشته‌باشند، تاخیر ارسال آن‌ها به سرور بالا رفته و تاثیری منفی بر روی تاخیر کل سیستم خواهدداشت. با توجه به اهمیت تاخیر پایین و بی‌درنگ بودن عملیات در این کاربرد، و همچنین دقت بالای بازه‌های R-R در تعیین نوع آریتمی، از این ویژگی استفاده کردیم. 
(normalized RR)

قدم بعد، پیاده‌سازی یک دسته‌بند برای تعیین نوع آریتمی است. در کارهای گذشته از الگوریتم‌های دسته‌بندی مانند SVM ANN (از رو مقاله هه بنویس با منبع) استفاده شده‌است. در کار پیش رو، SVM به دلیل کارایی مناسبی که در کارهای گذشته (منبع ۲۱ و ۲۲ مرده خوار) نشان داده‌است به کار گرفته شده‌است. 



%---------------------- begin chap 2 -------------
\cchapter{مفاهیم اولیه}
\label{chap:lit}
\pagebreak

\section {قلب و نحوه‌ی عملکرد آن}
قلب ماهیچه‌ای متشکل از ۴ حفره است. دو حفره‌ی بالایی، دهلیزهای چپ و راست نامیده می‌شوند و دو حفره‌ی پایینی، بطن‌های چپ و راست نام دارند. در هر سیکل تپش قلب، خونِ بدون اکسیژن از طریق بزرگ‌سیاهرگ‌های بالایی و پایینی وارد دهلیز راست می‌شود. پس از طی فرایندی در قلب، خون دارای اکسیژن شده و از بطن چپ خارج می‌شود. این خون سپس از طریق سرخرگ‌ها به اعضای بدن می‌رسد. قلب یک فرد بزرگسال سالم، به طور متوسط بین ۶۰ تا ۱۰۰ بار در دقیقه می‌تپد. \cite{MayoClinic}

عملکرد قلب توسط یک سیستم الکتریکی و به وسیله‌ی سیگنال‌های تولید شده در آن کنترل می‌شود. این سیگنال‌ها دیواره‌های قلب را تحریک می‌کنند و با انقباض دیواره‌ها، خون از قلب خارج شده و در سیستم گردش خون جریان می‌یابد. در ادامه به طور دقیق به نحوه‌ی عملکرد قلب می‌پردازیم. 


\subsection {سیستم هدایت الکتریکی قلب}
تمامی فعالیت‌های قلب که منجر به پمپ‌کردن خون در بدن می‌شوند، تحت کنترل سیستم هدایت الکتریکی قلب \LTRfootnote{Cardiac conduction system} قرار دارند. این سیستم با انتقال الکتریکی سیگنال‌های تولید شده، باعث به تپش درآمدن ماهیچه‌ی قلب می‌شود. بخش‌های اصلی این سیستم عبارت اند از:

\begin{enumerate}
	\item گره سینوسی‌دهلیزی \LTRfootnote{Sinoatrial node} \lr{(SA)} در دهلیز 
	راست قلب
	\item گره دهلیزی‌بطنی \LTRfootnote{Atrioventricular node} \lr{(AV)} در سپتوم داخل‌دهلیزی قلب \LTRfootnote {Interatrial septum} (دیواره‌ای ماهیچه‌ای که دهلیز راست و چپ قلب را جدا می‌کند)
	\item سیستم هیس-پورکینژ \LTRfootnote{His-Purkinje system} در دیواره‌های بطن‌های قلب
\end{enumerate}
این بخش‌ها در شکل \ref{fig:conduction} قابل مشاهده هستند.

\begin{figure}
\centering
\includegraphics[width=12cm]{Figures/conduction.jpg}
\caption{سیستم هدایت الکتریکی قلب\cite{Medicalexamprep}}
\label{fig:conduction}
\end{figure}

نقطه‌ی آغاز هر ضربان قلب، گره سینوسی‌دهلیزی است. این گره با تولید سیگنالی هر دو دهلیز را تحریک به انقباض می‌کند و در نتیجه‌ی این عمل، خون از طریق دریچه‌های باز، از دو دهلیز وارد دو بطن قلب می‌شود. سپس سیگنال وارد گره دهلیزی‌بطنی شده و برای لحظه‌ای کوتاه تاخیر می‌کند، تا خون فرصت پر کردن دو بطن قلب را پیدا کند. 

در مرحله‌ی بعد، سیگنال آزاد شده و در مسیری به نام دسته‌ی هیس \LTRfootnote{Hiss bundle} واقع در دیواره‌های بطن‌ها حرکت خود را ادامه می‌دهد. در این مرحله، سیگنال به دو دسته تقسیم شده و این دو دسته از طریق دو مسیر به نام‌های فیبرهای پورکینژ \LTRfootnote{Purkinje fibers} چپ و راست، به ترتیب وارد بطن چپ و راست قلب می‌شوند. این عمل باعث انقباض دو بطن می‌شود و در نتیجه‌ی این عمل، خون از طریق دریچه‌های بیرونی قلب، از آن خارج شده و به ریه‌ها و بقیه‌ی اعضای بدن انتقال می‌یابد. در این مرحله سیگنال از بطن‌ها گذر می‌کند و دو بطن وارد حالت استراحت می‌شوند، تا سیگنال بعدی فرابرسد.

تولید پی‌در‌پی این سیگنال‌ها، باعث انقباض و استراحت منظم و هماهنگ قلب شده و ضربان قلب را ایجاد می‌کند. در واقع ضربان قلب هر شخص، توسط تعداد دفعاتی در طول یک دقیقه که گره سینوسی‌دهلیزی سیگنال تولید می‌کند تعیین می‌شود. \cite{Heart} 


\section{آریتمی قلبی}
آریتمی قلبی به دسته‌ای از بیماری‌های قلبی اطلاق می‌شود که در آن‌ها، آهنگ تپش قلب حالتی غیرعادی پیدا می‌کند. به طور کلی دلیل رخ دادن آریتمی، عدم انتقال درست سیگنال‌های الکتریکی قلب بیان می‌شود. تعدادی از انواع آریتمی‌ها می‌توانند شدیدا خطرناک و کشنده باشند. اکثر آریتمی‌ها بی خطر شناخته شده‌اند، اما در صورت عدم تشخیص و رسیدگی به موقع می‌توانند زندگی عادی فرد مبتلا را آشفته ساخته یا حیات او را تهدید کنند. 
\subsection{انواع آریتمی قلبی}
آریتمی‌ها بر اساس نوع اختلالی که در ضربان قلب ایجاد می‌کنند، به چهار دسته‌ی کلی تقسیم می‌شوند.
\begin{enumerate}
	\item ضربان‌های زودرس \LTRfootnote{Premature beats}: در این دسته از آریتمی‌ها، قلب ضربا‌ن‌هایی زودرس تولید می‌کند که آهنگ طبیعی تپش آن را مختل می‌کنند. در صورتی که ضربان زودرس در بطن قلب تولید شده‌باشد، ضربان زودرس بطنی\LTRfootnote{Premature Ventricular Complex (PVC)}، و در صورتی که در دهلیز ایجاد شده باشد، ضربان زودرس دهلیزی \LTRfootnote{Premature Atrial Complex (AVC)} نامیده می‌شود.
	\item تاکی‌کاردی فوق بطنی \LTRfootnote{Supraventricular Tachycardia (SVT)}: در این نوع آریتمی، قلب به صورتی غیرعادی تندتر از معمول  (تقریبا بیش از ۱۰۰ ضربان در دقیقه) می‌تپد. \cite{Amboss} این آریتمی‌ها در بین گره سینوسی‌دهلیزی و گره دهلیزی‌بطنی ایجاد می‌شوند. 
	\item آریتمی‌های بطنی \LTRfootnote {Ventricular arrhythmia}:  آریتمی‌هایی که از پایین گره دهلیزی‌بطنی (در سطح بطن قلب) ریشه می‌گیرند در این دسته قرار دارند.
	\item برادی‌کاردی \LTRfootnote{Bradycardia}: در این نوع آریتمی، قلب بیمار آرام‌تر از حالت عادی می‌تپد و نرخ ضربان قلب معمولا پایین‌تر از ۶۰ تپش در دقیقه است.  \cite{Verywellhealth}

\end{enumerate}

\section{سیگنال نوار قلب}
همان طور که گفته شد، سلول‌های گره سینوسی تحریک الکتریکی منظمی را ایجاد می‌کنند که توسط سیستم هدایت الکتریکی موجود در قلب، به بخش‌های دیگر آن انتشار یافته و باعث تپش متناوب قلب می‌شود. نتیجه‌ی این فعالیت، ایجاد جریان الکتریکی در سطح بدن و تحریک تغییرات در پتانسیل الکتریکی سطح پوست است. این سیگنال‌ها را می‌توان به وسیله‌ی الکترودها و دیگر تجهیزات، ثبت و اندازه‌گیری نمود.

در فرایند ثبت نوار قلب، اختلاف پتانسیل بین نقاط قرارگیری الکترودها بر روی بدن اندازه‌گیری شده و معمولا به کمک تقویت‌کننده‌های عملیاتی \LTRfootnote{Operational amplifiers} بهبود داده می‌شود. در مرحله‌ی بعد، سیگنال ابتدا از یک فیلتر بالاگذر و سپس از یک فیلتر پایین‌گذر تصحیح فرکانس عبور داده‌می‌شود. در نهایت این سیگنال آنالوگ، به سیگنال دیجیتال تبدیل می‌شود. منحنی گرافیکی رسم شده در انتهای این فرایند، نوار قلب، و یا به اختصار \lr{ECG} نامیده می‌شود. 

امروزه در روش‌های استاندارد اندازه‌گیری نوار قلب،تعدادی الکترود بر روی سطح پوست قرارمی‌گیرند و یکی از آن‌ها به عنوان مرجع \LTRfootnote{Reference} برای دیگر الکترودها در نظر گرفته می‌شود. به طور معمول، الکترود مرجع روی ساق پای راست نصب می‌شود. \cite{ECGSurvey} هر یک از الکترودهای دیگر، ولتاژ ناحیه‌ی قرارگیری خود را نسبت به ولتاژ الکترود مرجع اندازه‌گیری می‌کنند. هر یک از این اختلاف پتانسیل‌های اندازه‌گیری شده، یک لید \LTRfootnote{Lead} نامیده می‌شود. 
\subsection{نحوه‌ی قرارگیری الکترودها بر روی پوست و لیدهای تولیدشده}

یکی از ترکیب‌های رایج قراردادن الکترودها متشکل از ۱۰ الکترود است که بر روی دست، پا و سینه‌ی بیمار قرار می‌گیرند. از ترکیب این الکترودها ۱۲ لید ایجاد می‌شود که به سه دسته‌ی کلی تقسیم می‌شوند:
\begin{itemize}
	\item  سه لید دوقطبی اندامی \LTRfootnote{Bipolar limb leads} به نام‌های \lr{I}، \lr{II} و \lr{III}
	\item سه لید تک‌قطبی اندامی \LTRfootnote{Unipolar limb leads} به نام‌های \lr{aVF}، \lr{aVL} و \lr{aVR}
	\item شش لید تک‌قطبی سینه‌ای به نام‌های \lr{V1} تا \lr{V6}
\end{itemize}
  هر یک از این لیدها فعالیت الکتریکی قلب را از یک زاویه‌ی خاص در بدن نشان می‌دهد. پرکاربردترین لید برای تشخیص بیماری‌های قلبی، لید \lr{II} می‌باشد که اختلاف پتانسیل بین الکترودهای ساق پای چپ و بازوی راست را نشان می‌دهد. در شکل \ref{fig:leads} یک نوار قلب ۱۲ لیدی مشاهده می‌شود. منحنی رسم شده از هر لید به صورت جداگانه نشان داده شده‌است و لید \lr{II} نیز به تنهایی رسم شده‌است. این لید به خصوص از آن جهت اهمیت دارد که نمای خوبی از ترکیب \lr{QRS}
ارائه می‌دهد. در بخش بعد در مورد این موضوع به تفصیل توضیح داده خواهد شد.
\begin{figure}
\centering
\includegraphics[width=16cm]{Figures/leads.png}
\caption{نوار قلب ۱۲ لیدی گرفته‌شده از یک فرد سالم\cite{Drsmith}}
\label{fig:leads}
\end{figure}

\subsection{ترکیب QRS}
با بررسی یک سیکل ضربان قلب در نوار قلب، ۵ انحراف\LTRfootnote{Deflection}
یا موج پراهمیت دیده می‌شود. اولین موج، \lr{P} نام دارد که با فعال شدن دهلیزهای راست و چپ و بالارفتن پتانسیل الکتریکی آن‌ها اتفاق می‌افتد. سه موج بعدی به ترتیب \lr{Q}، \lr{R} و \lr{S} نام دارند. این سه موج به ترتیب و با فاصله‌ی کمی از هم رخ می‌دهند و عموما به عنوان یک ترکیب، همراه یکدیگر بررسی می‌شوند. این ترکیب که \lr{QRS} نامیده می‌شود، واضح‌ترین بخش مشاهده‌شده در یک سیکل قلبی است که مدت زمان بالارفتن پتانسیل ماهیچه‌های بطنی قلب را نشان می‌دهد. موج بعدی \lr{T} نام دارد که در طول آن بطن‌ها منقبض شده و بار مثبت خود را تخلیه می‌کنند. ترکیب \lr{QRS} در شکل \ref{fig:QRS} مشاهده می‌شود.

\begin{figure}
\centering
\includegraphics[width=10cm]{Figures/qrs.png}
\caption{ ترکیب \lr{QRS} \cite{Miramontes2017}}
\label{fig:QRS}
\end{figure}
 
\subsubsection{بازه‌های زمانی مهم در سیکل ضربان قلب}
مهم‌ترین بازه‌های زمانی در یک سیکل ضربان قلب عبارت اند از:
\begin{itemize}
	\item بازه‌ی \lr{PR}: فاصله‌ی زمانی از ابتدای موج \lr{P} تا ابتدای ترکیب QRS 
	\item مدت‌زمان \lr{QRS}: مدت‌زمان رخ‌دادن ترکیب \lr{QRS}
	\item بازه‌ی \lr{QT}: فاصله‌ی زمانی از ابتدای ترکیب \lr{QRS} تا انتهای موج \lr{T}
	\item بازه‌ی \lr{RR}: مدت‌زمان سیکل کامل قلب که نشان‌دهنده‌ی سیکل کامل بطن‌ها می‌باشد. 
	\item بازه‌ی \lr{PP}: مدت‌زمان سیکل کامل دهلیزی
\end{itemize}


 \subsubsection{تاثیر آریتمی قلبی بر روی شکل ترکیب QRS}
 وجود آریتمی قلبی می‌تواند باعث تغییر شدید در امواج \lr{Q}، \lr{R} و \lr{S} شود. لید \lr{II} به دلیل واضح‌تر نشان دادن ترکیب \lr{QRS} و لیدهای \lr{V1} تا \lr{V6} به دلیل این که الکترودهای آن‌ها بر روی سینه قرارگرفته و تشخیص بهتر تغییرات پتانسیل ماهیچه‌ی بطنی را ممکن می‌سازند، تا کنون بهترین نتایج را در تشخیص آریتمی نشان داده‌اند. \cite{ECGSurvey}

در طول بازه‌ی زمانی \lr{QRS} بطن‌هابه وسیله‌ی سیستم هیس-پورکینژ منقبض می‌شوند. این سیستم شامل سلول‌هایی در دیواره‌های بطن‌ها است که خاصیت رسانایی سریع الکتریکی را دارند. در صورت ایجاد اختلال در کار این سیستم و ضعیف‌شدن خاصیت رسانایی الکتریکی سلول‌ها، بازه‌ی زمانی \lr{QRS} طولانی‌تر می‌شود. در برخی موارد سیگنال الکتریکی به جای انتقال یافتن از طریق سیستم هیس-پورکینژ، از طریق ماهیچه‌های قلب منتقل می‌شود. این اتفاق منجر به طولانی شدن زمان انتقال الکتریکی سیگنال و در نتیجه عریض شدن بازه‌ی \lr{QRS} می‌شود.
 به طور معمول طول یک بازه‌ی \lr{QRS} بین ۰/۰۸ تا ۰/۱ ثانیه است. در مواردی که طول این بازه از ۰/۱۲ ثانیه بیشتر شود، \lr{QRS} غیرعادی تلقی می‌شود. \cite{Healio}

\section{مسائل دسته‌بندی}

در مسائل دسته‌بندی، ورودی‌های مسئله تعدادی داده هستند و مطلوب مسئله، جای دادن هر یک از داده‌ها در یک دسته یا کلاس است. به بیان رسمی‌تر در این مسئله‌ها، هدف، تخمین‌زدن یک نگاشت از متغیرهای ورودی \lr{X} به تعدادی متغیر خروجی گسسته \lr{Y} است. این متغیرهای خروجی تعدادی برچسب\LTRfootnote{Label} هستند که تعیین می‌کنند هر داده در کدام دسته قرار می‌گیرد. تعداد این دسته‌ها می‌تواند دو و یا بیشتر باشد که در حالت دوم، مسئله یک مسئله‌ی دسته‌بندی چنددسته‌ای\LTRfootnote{Multiclass classification problem} نامیده می‌شود. 

\subsection{روش ماشین بردار پشتیبانی (SVM)}
یکی از پرکاربردترین دسته‌های الگوریتم برای حل مسایل دسته‌بندی، الگوریتم‌های \lr{SVM} هستند. در این الگوریتم‌ها، داده‌ها به مثابه‌ی نقطه‌هایی در یک فضای \lr{N}بعدی فرض می‌شوند. هدف الگوریتم، یافتن ابرصفحه‌هایی\LTRfootnote{Hyperplanes} است که به طور بهینه نقطه‌های داده‌ها را به کلاس‌های متعدد دسته‌بندی کند. تعداد بعدهای این فضا \lr{(N)} برابر با تعداد ویژگی‌ها است. معمولا تعداد زیادی ابرصفحه را می‌توان برای جداسازی دو کلاس مختلف از داده‌ها یافت، اما در این الگوریتم، هدف یافتن ابرصفحه‌ای است که بیشترین فاصله را با نزدیک‌ترین نقطه‌ی داده در هر یک از کلاس‌ها داشته‌باشد. این فاصله، حاشیه\LTRfootnote{Margin} نامیده می‌شود. 

\subsubsection{ابرصفحه}
ابرصفحه مرزی است که نقاط داده‌ها را در یک فضای \lr{N}بعدی به دو بخش تقسیم می‌کند. برای مثال در مسئله‌ای با دو کلاس هدف، نقاطی که در هر یک از دو سمت ابرصفحه‌ی به‌دست‌آمده قرار می‌گیرند، به یکی از آن دو کلاس تعلق می‌یابند. تعداد بعد ابرصفحه بسته به تعداد ویژگی‌های داده‌ها است. مثلا در مسئله‌ای که سه ویژگی برای داده‌ها به دست آورده‌ایم، فضای داده ۳بعدی بوده و در نتیجه ابرصفحه‌ی جداکننده‌ی داده‌ها نیز ۳بعدی خواهد بود.

\subsubsection{بردار پشتیبانی}
بردارهای پشتیبانی، نقاط داده‌ای هستند که ابرصفحه را تعریف می‌کنند. این نقاط به ابرصفحه نزدیک‌تر بوده و بر روی موقعیت قرارگیری و جهت آن تاثیر می‌گذارند. به کمک این بردارها، ابرصفحه‌ای با بیشترین حاشیه برای دسته‌بندی انتخاب می‌شود.\cite{SVM} نمودار یک مسئله‌ی دسته‌بندی دوبعدی در شکل \ref{fig:SVMClassification} دیده می‌شود.

\begin{figure}
\centering
\includegraphics[width=10cm]{Figures/svmMargin.png}
\caption{ نموداری از حل یک مسئله‌ی دسته‌بندی دوبعدی با روش \lr{SVM}\cite{SVM}}
\label{fig:SVMClassification}
\end{figure}
 

\subsubsection{ تابع کرنل}
در روش \lr{SVM} برای دسته‌بندی داده‌ها از توابعی به نام توابع کرنل استفاده می‌شود. تابع کرنل داده را به عنوان ورودی گرفته و آن را به فضایی دیگر انتقال \LTRfootnote{Transform} می‌دهد. به کمک تابع کرنل، داده‌هایی که در فضای عادی مشاهده شده‌اند، به فضایی با تعداد ابعاد بالاتر انتقال می‌یابند که در چنین فضایی امکان جداسازی آن‌ها وجود دارد. در واقع هر مدل خطی را می‌توان به کمک تابع کرنل به یک مدل غیر خطی تبدیل کرد، به این صورت که ویژگی‌های مدل را با یک تابع کرنل جایگزین کنیم. 

به طور رسمی‌تر می‌توان تابع کرنل را به این صورت تعریف کرد: به ازای هر  $x$  و $x'$ در فضای $X$ می‌توان توابعی به صورت $k(x, x')$ را یافت که حاصل ضرب داخلی دو نقطه در فضای دیگری به نام $V$ است. این روابط در معادله‌ی \ref{eq:kernel} قابل مشاهده است.\cite{KernelSVM} 

\begin{equation}
\begin{split}
	& k: X \times X \to \mathbb{R} \\
	& k(x_i, x_j) = \bigg \langle \Phi (X_i), \Phi(X_j) \bigg \rangle
\end{split}
\label{eq:kernel}
\end{equation}

ساده‌ترین نوع کرنل، کرنل خطی است. این توابع  داده‌ها را به فضایی با تعداد بعد بالاتر نگاشت نمی‌کنند، به همین دلیل بهتر است در مسائلی که داده‌ها به صورت خطی قابل جداسازی هستند، از این نوع کرنل استفاده شود. 
 این نوع کرنل‌ها به دلیل سادگی و خطی بودن، سرعت بیشتری در دسته‌بندی دارند. معمولا در مسائلی که تعداد ویژگی‌ها زیاد بوده و نگاشت داده‌ها به نقاطی در فضای با تعداد بعدهای بالاتر تاثیر چشمگیری در بهبود دسته‌بندی ندارد، از کرنل خطی استفاده می‌شود. \cite{LinearSVM} 
 نوع پیچیده‌تری از کرنل که در بسیاری از مسائل دسته‌بندی کاربرد دارد. کرنل \lr{RBF}\LTRfootnote{ Radial Basis Function} نام دارد. این تابع بر روی نقطه‌ی $X_i$ و $X_j$ در فضای $X$ که یک فضای ورودی است در معادله‌ی \ref{eq:kernelRBF} قابل مشاهده است.\cite{kernelRBF}
   
\begin{equation}
	 k(X_i, X_j) = \exp(-\frac{{||X_i-X_j||}^2}{2\sigma^2})
\label{eq:kernelRBF}
\end{equation}
در این رابطه $\sigma$ یک پارامتر آزاد است. این تابع، دو بردار $X_i$  و $X_j$ که در فضایی دو بعدی قرار دارند را به یک بردار بی‌نهایت نگاشت می‌کند. این عمل باعث می‌شود نقاط داده به نقاطی در فضایی با تعداد بعد بیشتر نگاشت شوند. در مسائلی که در فضای اصلی داده‌های ورودی، ابرصفحه‌ای برای جداسازی کلاس‌ها یافت نمی‌شود، می‌توان با استفاده از کرنل \lr{RBF} در فضایی با تعداد بعد بالاتر، ابرصفحه‌ای برای جداسازی کلاس‌ها یافت. این موضوع در شکل \ref{fig:RBFKernel} قابل مشاهده است. این نوع تابع کرنل، زمان و قدرت پردازشی بیشتری به نسبت کرنل خطی مصرف می‌کند.

\begin{figure}
\centering
\includegraphics[width=15cm]{Figures/RBFKernel.png}
\caption{ سمت چپ: داده‌های غیر قابل جداسازی توسط یک ابرصفحه در یک فضای دوبعدی، سمت راست: داده‌های انتقال‌داده‌شده به فضای سه‌بعدی و قابل جداسازی\cite{TowardsScienceRBF}}
\label{fig:SVMClassification}
\end{figure}

\subsubsection{حل مسائل دسته‌بندی  با استفاده از تابع کرنل RBF}


\subsection{دسته‌بندی داده‌ها با استفاده از روش SVM}


%------------------------ Start chapter 3 -------------
\cchapter{روش حل مسئله}
\pagebreak

\section{مقدمه}
 این پروژه در دو بخش کلی پیش‌پردازش در سمت سخت‌افزار و پردازش در سرور انجام شده‌است. در بخش اول، تعدادی پردازش اولیه بر روی داده‌های خام ضربان قلب انجام می‌شود. این بخش یک بستر پیاده شده بر روی سخت‌افزار است که برای کامل شدن باید به یک سنسور ضربان قلب متصل شود. این بخش همراه بیمار خواهد بود و پردازش‌های ساده‌ی اولیه را بر روی سیگنال نوار قلب انجام خواهد داد و نتایج آن به سرور ارسال می‌شود.  پردازش‌های پیچیده‌تر برای تشخیص آریتمی بر عهده‌ی سرور خواهد بود. در سرور یک الگوریتم دسته‌بندی بر روی داده‌ا انجام شده و کلاس آریتمی آن‌ها تشخیص داده می‌شود.

\section{عملیات پیش‌پردازش بر روی سخت‌افزار} 
در این بخش عملیات پیش‌پردازش با هدف تشخیص ترکیب QRS در هر ضربان قلب بر روی سیگنال دیجیتال ضربان قلب اجرا می‌شود. خروجی این عملیات، موقعیت زمانی قله‌ی R در ترکیب QRS هر ضربان است که در پردازش‌های آینده برای تشخیص آریتمی آن ضربان مورد استفاده قرار می‌گیرد.

	\subsection{مراحل تشخیص QRS}
	
	پیش از ورود سیگنال نوار قلب به ماژول پیش‌پردازش، نوار قلب خام گرفته‌شده از بیمار  از یک مبدل آنالوگ به دیجیتال\LTRfootnote{ADC} عبور کرده و با نرخ نمونه‌برداری\LTRfootnote{Sampling rate} معینی به سیگنال دیجیتال تبدیل می‌شود. مقدار این نرخ نمونه‌برداری در برخی مراحل پیش‌پردازش اهمیت دارد. 
پس از دیجیتال شدن، سیگنال وارد ماژولی که برای تشخیص \lr{QRS} طراحی کرده‌ایم می‌شود. در ادامه به مراحل اصلی طی‌شده در این بخش می‌پردازیم.
		\subsubsection{حذف نویز سیگنال به کمک فیلتر میان‌گذر}
		اولین مرحله در تشخیص \lr{QRS} حذف نویز سیگنال نوار قلب است. در حین ثبت ضربان قلب، منابع مختلفی از نویز در سیگنال اختلال ایجاد می‌کنند.در یک سیگنال ECG به طور معمول نویزهای فرکانس پایینی ناشی از baseline wander وجود دارد. این نویزها به علت حرکت الکترودها بر روی پوست و همین طور اعمالی چون حرکات و تنفس بیمار به وجود می‌آیند. انقباض ماهیچه‌های اطراف قلب نیز یکی دیگر از منابع نویز است. این انقباضات توسط الکترودها ثبت شده و در نوار قلب نویزهای فرکانس بالایی ایجاد می‌کنند.\cite{Joshi2013}
		
		با توجه به نویزهای معمول، محدوده‌ی فرکانسی مطلوب برای بیشینه‌کردن انرژی QRS و کمینه‌کردن انرژی نویز، ۵ تا ۱۵ هرتز تشخیص داده شده‌است.\cite{Pan1985} به منظور نگه‌داشتن این بازه‌ی فرکانسی و حذف فرکانس‌های بالا و پایین آن، سیگنال دیجیتال از یک فیلتر میان‌گذر عبور داده می‌شود. این فیلتر متشکل از یک فیلتر پایین‌گذر و یک فیلتر بالاگذر متوالی است. هر دوی این فیلترها به صورت نرم‌افزاری پیاده‌سازی شده‌اند. هر دوی این فیلترها، IIR بوده و زمان‌گسسته هستند. 
تابع تبدیل فیلتر پایین‌گذر را در معادله‌ی \ref{eq:lowpassTr} مشاهده می‌کنیم.

\begin{equation}
	H(z) = \frac{{(1-z^{-6})}^2}{{(1-z^{-1})}^2}
\label{eq:lowpassTr}
\end{equation}
	
	معادله‌ی تفاضلی این فیلتر به صورت معادله‌ی \ref{eq:lowpassDE} در خواهد آمد.
	
\begin{equation}
	y(nT) = 2y(nT-T) - y(nT-2T) + x(nT) - 2x(nT-6T) + x(nT-12T) 
\label{eq:lowpassDE}
\end{equation}
فرکانس قطع این فیلتر پایین‌گذر ۱۱ هرتز و gain آن ۳۶ است. یک فیلتر بالاگذر به صورت سری با این فیلتر قرار می‌گیرد که تابع تبدیل آن به صورت معادله‌ی \ref{eq:highpassTr} است.
\begin{equation}
	H(z) = \frac{{(-1+32z^{-16}+z^{-32})}}{{(1+z^{-1})}}
\label{eq:highpassTr}
\end{equation}
که معادله‌ی تفاضلی آن به صورت معادله‌ی \ref{eq:highpassDE} خواهد بود.
 \begin{equation}
	y(nT) = 32x(nT-16T) - [y(nT-T) + x(nT) - x(nT-32T)]
\label{eq:highpassDE}
\end{equation}
این فیلتر فرکانس‌های بالای ۵ هرتز را عبور می‌دهد و gain آن ۳۲ است. از توالی این دو فیلتر، فیلتر میان‌گذری به دست می‌آید که فرکانس‌های ۵ تا۱۱ هرتز را عبور می‌دهد که به هدف ما برای کاهش نویز نزدیک است. 

\subsubsection{مشتق‌گیر}
پس از اعمال فیلترها، عمل مشتق‌گیری بر روی سیگنال انجام می‌شود. مشتق‌گیری از سیگنال، اطلاعاتی در مورد شیب آن در بازه‌ی QRS فراهم می‌کند. تابع انتقال این فیلتر به صورت معادله‌ی \ref{eq:derivativeTr} است و معادله‌ی تفاضلی آن به صورت رابطه‌ی \ref{eq:derivativeDE} می‌آید.

\begin{equation}
	H(z) = \frac{(-z^{-2}-2z^{-1}+2z+z^2)}{8T}
\label{eq:derivativeTr}
\end{equation}
	
\begin{equation}
	y(nT) = \frac{-x(nT-2T)-2x(nT-T)+2x(nT+T)+x(nT+2T)}{8T}
\label{eq:derivativeDE}
\end{equation}

\subsubsection{مجذورکننده}
پس از مشتق‌گیری، مجذور سیگنال به صورت نقطه به نقطه به دست می‌آید. معادله‌ی تفاضلی فیلتر در این بخش به صورت معادله‌ی \ref{eq:square} است. اعمال این فیلتر بر روی خروجی مشتق‌گیر، باعث می‌شود تمامی نقاط سیگنال مثبت شده و به دلیل انجام عمل مربع‌کردن، فواصل نقاط گسسته‌ی سیگنال تشدید شود.

\begin{equation}
	y(nT) = [x(nT)]^2
\label{eq:derivativeDE}
\end{equation}

\subsubsection{انتگرال‌گیر با پنجره‌ی لغزان}
در این مرحله سیگنال مربع‌شده وارد یک انتگرال‌گیر می‌شود. هدف از این کار، به دست آوردن اطلاعاتی در مورد شکل موج سیگنال، علاوه بر اطلاعات مربوط به شیب موج R است که در مراحل قبل به دست آمد. معادله‌ی تفاضلی این انتگرال‌گیر به صورت معادله‌ی \ref{eq:integrator} است.

\begin{equation}
	y(nT) = \frac{x(nT-(N-1)T) + x(nT-(N-2)T+...+x(nT))}{N}
\label{eq:integrator}
\end{equation}

که در آن N تعداد نمونه‌ها در طول پنجره‌ی انتگرال‌گیر است. $N$ به صورت تجربی به دست می‌آید و در تشخیص نهایی \lr{R} اهمیت زیادی دارد. به طور معمول $N$ باید تقریبا به اندازه‌ی عریض‌ترین بازه‌ی \lr{QRS} باشد. در صورتی که پنجره بیش از حد عریض باشد، در هنگام انتگرال‌گیری، شکل موج \lr{QRS} با موج \lr{T} ترکیب می‌شود. اگر پنجره بیش از حد کوتاه باشد، کل بازه‌ی \lr{QRS} را در بر نمی‌گیرد و در این بازه تعداد زیادی قله تولید خواهد شد. این مقدار به طور تجربی به دست آمده و با نرخ نمونه‌برداری ارتباط دارد. در این پروژه طول پنجره ۷۰ در نظر گرفته شده‌است.

\subsubsection{تعیین موقعیت قله‌های R با کمک مقدارهای آستانه}

موج \lr{QRS} هم‌زمان با لبه‌ی بالارونده‌ی انتگرال‌گیر رخ می‌دهد، و طول بازه‌ی این لبه برابر با طول بازه‌ی \lr{QRS} است. به این ترتیب، می‌توان موقعیت زمانی \lr{QRS} را از روی جایگاه لبه‌ی بالارونده تعیین کرد. با استفاده از این اطلاعات، و همین طور اطلاعات مربوط به شیب منحنی \lr{QRS} در این بازه، می‌توان نقطه‌ی ثابتی را به عنوان موقعیت قله‌ی \lr{R} به دست آورد.
برای تعیین درست موقعیت قله‌ی \lr{R} تعدادی ولتاژ آستانه\LTRfootnote{Threshold}اعمال می‌شوند و به نسبت بالاتر یا پایین‌تر بودن ولتاژ هر نمونه از آن‌ها، وجود یا عدم وجود قله تشخیص داده می‌شود. این آستانه‌ها با گذشت زمان با نویز تطبیق می‌یابند. در مجموع دو سری ولتاژ آستانه داریم که هر کدام شامل دو آستانه هستند. در هر یک از این دو سری، آستانه‌ی بالاتر برای تحلیل اولیه‌ی سیگنال استفاده می‌شود، و در صورتی که در یک بازه‌ی زمانی مشخص \lr{QRS} ای تشخیص داده نشده باشد، لازم است در این بازه از تکنیک جستجوی برگشتی\LTRfootnote{Search-back} استفاده شود. در این تکنیک در این بازه‌ی زمانی  از آستانه‌های پایین‌تر برای تشخیص \lr{QRS} استفاده می‌شود. روابط این آستانه‌ها در معادله‌ی \ref{eq:thresholds} مشاهده می‌شوند. در این روابط، $PEAK1$ بالاترین ولتاژ سیگنال به طور کلی، $SPKI$ تخمین جاری از بالاترین ولتاژ سیگنال و $NPKI$ تخمین جاری از بالاترین ولتاژ نویز در هر لحظه است. همچنین $THRESHOLD I1$ اولین مقدار آستانه‌ی اعمال‌شده بر روی سیگنال انتگرال‌گیری‌شده و $THRESHOLD I2$ دومین مقدار آستانه و نصف مقدار آستانه‌ی اول است.

\begin{align}
\begin{split}
	& SPKI = 0.125 PEAKI + 0.875 SPKI\\
	& NPKI = 0.125 PEAKI + 0.875 NPKI\\
	& THRESHOLD I1 = NPKI + 0.25(SPKI - NPKI)\\
	& THRESHOLD I2 = 0.5 THRESHOLD I1\\
\end{split}
\label{eq:integrator}
\end{align}
برای این که یک نمونه به عنوان قله‌ی \lr{R} تشخیص داده شود، باید مقداری بالاتر از $THRESHOLD I1$ داشته باشد. در صورتی که یک قله‌ی \lr{R} در فرایند جستجوی برگشتی تشخیص داده شود، مقدار $SPK_I$ به صورت رابطه‌ی \ref{eq:thresholdSPKI} به‌روز خواهد شد. 
\begin{equation}
	SPKI = 0.25 PEAKI + 0.75 SPKI
\label{eq:thresholdSPKI}
\end{equation}

	\subsection{پیاده‌سازی الگوریتم تشخیص QRS بر روی بستر سخت‌افزاری}
ورودی این بخش، سیگنال دیجیتال دریافت شده از سنسور ضربان قلب است. نحوه‌ی تولید این سیگنال و نوع سنسور به‌کاررفته برای آن کاملا به کاربرد بستگی داشته و در این پروژه تاکیدی بر روی آن نیست. محاسبات انجام‌شده در الگوریتم تشخیص \lr{QRS}، به مقدار نرخ نمونه‌برداری سیگنال ضربان قلب وابسته است. پارامترهای الگوریتم پیاده‌سازی شده در این بخش، برای نرخ نمونه‌برداری ۳۶۰ نمونه بر ثانیه بهینه شده‌اند و از این روی، لازم است نرخ نمونه‌برداری سیگنال دیجیتال ورودی، مساوی با ۳۶۰ یا نزدیک به آن باشد.

خروجی این بخش، موقعیت زمانی قله‌ی \lr{R} در هر یک از بازه‌های \lr{QRS} تشخیص‌داده‌شده در ضربان قلب است. به بیان دیگر، الگوریتم برخی از نمونه‌ها در سیگنال را به عنوان قله‌ی \lr{R} تشخیص داده و شماره‌ی آن نمونه را به عنوان خروجی برمی‌گرداند. این مقادیر باید برای انجام پردازش‌های آینده به سرور ارسال شوند. از آن‌جا که از کل سیستم انتظار بی‌درنگ‌بودن داریم، علاوه بر تشخیص بی‌درنگ \lr{QRS} لازم است دریافت داد‌ه‌های خام از حسگر و همین‌طور فرستادن قله‌های lr{R} تشخیص‌داده‌شده به سرور نیز به صورت بی‌درنگ و در حین تشخیص \lr{QRS} انجام شود. به بیان بهتر، در چنین کاربردی انجام تشخیص \lr{QRS} بر روی ضربان قلب به طور کامل و سپس فرستادن تمامی Rهای تشخیص‌داده‌شده به سرور قابل قبول نخواهد بود.
کارهای انجام‌شده در این بخش را می‌توان در قالب موارد زیر بیان کرد. 
\subsubsection{دریافت داده‌های خام جدید از حس‌گر}
در این بخش، هدف بر این است که رفتار یک حس‌گر دیجیتال ضربان قلب با نرخ نمونه‌برداری ۳۶۰ نمونه بر ثانیه شبیه‌سازی شود. بهترین راه‌حل برای این کار، استفاده از ارتباط سریال بین ماژول و یک رایانه (به جای حس‌گر) تشخیص داده شد. با فرض این که داده‌های چنین حس‌گری قبلا دریافت و بر روی رایانه ذخیره شده باشد، در صورتی که در هر ثانیه ۳۶۰ نمونه از رایانه به ESP ارسال کنیم، رفتار یک حس‌گر دیجیتال با نرخ نمونه‌برداری ۳۶۰ را شبیه‌سازی کرده‌ایم.

پیاده‌سازی این بخش به این صورت انجام شد که پایه‌های RX و TX ماژول ESP به پورت سریال یک کامپیوتر وصل شد و داده‌های دیجیتال ضربان قلب که قبلا به وسیله‌ی یک سنسور دیجیتال تولید شده بودند،‌ به وسیله‌ی اسکریپتی در کامپیوتر به ESP ارسال شدند. در هر ثانیه ۳۶۰ مقدار از مقادیر ذخیره شده با نرخ باد ۱۱۲۵۰۰ بیت بر ثانیه به ESP ارسال شدند. ESP این داده‌ها دریافت کرده و پردازش‌های آینده را بر روی آن‌ها انجام خواهد داد. این ماژول به طور دائم در حال اجرای الگوریتم تشخیص QRS بر روی داده‌هایی که قبلا دریافت کرده است می‌باشد، و در این حین داده‌های جدیدی نیز از سمت رایانه (حس‌گر) دریافت می‌کند.
\subsubsection{اعمال الگوریتم و فرستادن شماره‌ی نمونه به  سرور
 در صورت تشخیص قله}
	 هدف این بخش این است که  ماژول \lr{ESP8266} الگوریتم تشخیص QRS را بر روی نمونه‌هایی که دریافت می‌کند اجرا کرده و در صورت تشخیص قله، موقعیت زمانی آن را برای سرور بفرستد. در همین حین، هر لحظه نمونه‌های جدیدی از طریق ارتباط سریال دریافت می‌شوند. چالش به‌وجودآمده در این مرحله این است که این نمونه‌های جدید نباید از دست بروند. یک راه حل ممکن برای این موضوع،‌ پیاده‌سازی نوعی مکانیزم چندنخی\LTRfootnote{Multithreading} در \lr{ESP8266} است. در یکی از نخ‌ها، داده‌های جدید دریافت شوند و در نخ دیگر الگوریتم بر روی داده‌‌های موجود اجرا شود.
	 
با بررسی‌های انجام‌شده دریافت شد که پیاده‌سازی چندنخی بر روی \lr{ESP8266} پیچیدگی بالایی داشته و کارا نمی‌باشد. به جای پیاده‌سازی این روش، از امکان ایجاد وقفه‌ی سریال در هنگام دریافت داده استفاده شد. \lr{ESP8266} امکان دریافت داده‌ها به صورت مبتنی بر وقفه را دارد، که در کتاب‌خانه‌ی \lr{HardwareSerial} به طور کامل پیاده‌سازی شده است. نحوه‌ی پیاده‌سازی به این شکل است که به محض ورود داده‌ی سریال جدید، \lr{ESP8266} کار خود را رها کرده و به وقفه سرویس می‌دهد. در روتین وقفه، کاراکتر تازه وارد از طریق ارتباط سریال، در بافر سریال \lr{ESP8266} می‌شود. سپس برنامه از روتین وقفه خارج شده و به ادامه‌ی کار خود باز می‌گردد. با استفاده از این امکان \lr{ESP8266} قادر است به طور همزمان با اجرای الگوریتم، نمونه‌های جدید را دریافت کند. به دلیل محدود بودن حجم بافر سریال داخلی موجود در \lr{ESP8266}، نیاز به پیاده‌سازی یک مکانیزم بافرینگ در خود کد نیز وجود دارد. برای جلوگیری از سرریز کردن بافر سریال، در ابتدای هر لوپ اجرای برنامه‌ی \lr{ESP8266} به این بافر سرکشی شده و داده‌های جدید را از آن بر می‌داریم و در بافری که خود پیاده‌سازی کرده‌ایم قرار می‌دهیم. این بافر برای اطمینان حجم بیشتری دارد و با استفاده از آرایه پیاده‌سازی شده‌است. داده‌های جدید در این آرایه می‌مانند، تا وقتی که نوبت پردازش و انجام الگوریتم روی آن‌ها فرا برسد. در این جا از \lr{Thingspeak API} به عنوان فضایی برای ذخیره‌ی این داده‌ها استفاده شد.
	

\section{عملیات پردازش سمت سرور}

	\subsection{نحوه‌ی دریافت داده‌های پیش‌پردازش شده در سرور}
	در کد سمت سرور،‌ داده‌ها از \lr{Thingspeak API} به صورت بی‌درنگ دریافت می‌شوند. در این جا منظور از بی‌درنگ بودن این است که برنامه‌ی سمت سرور دائما در یک حلقه به \lr{API} درخواست داده و داده‌هایی که در فاصله‌ی این درخواست و درخواست قبلی در سرور ثبت شده‌اند را دریافت می‌کند. همان‌طور که اشاره شد، هر کدام از این داده‌ها شماره‌ی نمونه‌ی یک قله‌ی \lr{R} هستند. داده‌هایی که در هر نوبت خواندن از \lr{API} دریافت شده‌اند، در یک آرایه ذخیره می‌شوند. سپس عملیات استخراج ویژگی‌ها روی هر یک از این داد‌ه‌ها اجرا می‌شود تا ویژگی‌های هر داده برای ورود به مدل \lr{SVM} و انجام عملیات دسته‌بندی آماده شود.


	\subsection{داده‌های مورد بررسی در الگوریتم یادگیری}

		\subsubsection{پایگاه داده‌ی MIT-BIH}

		\subsubsection{نحوه‌ی تقسیم داده‌ها به دو مجموعه‌ی آموزش و تست}

	\subsection{نحوه‌ی اجرای الگوریتم یادگیری}
		
		\subsubsection{استخراج ویژگی‌ها}
		
		\subsubsection{پارامترهای به کار گرفته‌شده در الگوریتم}
		
		\subsubsection{استراتژی رای‌دهی}

	\subsection{ارزیابی نتایج حاصل از یادگیری}
	
		\subsubsection{معیارهای کارایی}

%------------------------ End chapter 3 -------------


\cchapter{نتایج به‌دست‌آمده }
\label{chap:result}
\pagebreak

در این کار، آزمایش‌هایی با هدف بهینه کردن زمان پاسخ کل الگوریتم و دقت الگوریتم دسته‌بندی انجام شد. هم‌چنین برای ارتباط بهتر کاربران (بیمار و پزشک) با سیستم، یک اپلیکیشن تحت وب برای دسترسی کاربر به نتایج دسته‌بندی طراحی و پیاده‌سازی شد. در ادامه به نتایج به‌دست‌آمده در این بخش‌ها می‌پردازیم.

\section{زمان پاسخ سیستم}
در این کاربرد، زمان پاسخ را مدت‌زمان بین تولید یک ضربان قلب در سیگنال نوار قلب، تا لحظه‌ای که کلاس آن ضربان به کاربر نشان داده می‌شود در نظر گرفته‌ایم. این زمان معیاری برای بررسی سرعت و میزان کارآمدبودن سیستم، به عنوان یک سیستم بی‌درنگ بوده و از این جهت اهمیت بالایی دارد. 

همان‌طور که پیش‌تر توضیح داده شد،در ابتدا سخت‌افزار با دریافت یک ضربان قلب، الگوریتم تشخیص \lr{QRS} را بر روی آن اجرا کرده و قله‌ی \lr{R} را در ضربان تشخیص می‌دهد. مدت‌زمانی که طول می‌کشد تا این عمل انجام شود را $t_{QRS}$ نامیده‌ایم. این زمان به طور متوسط ۳۲۰ میکروثانیه محاسبه شد که به دلیل ناچیزبودن در برابر زمان‌های محاسبه‌شده‌ی دیگر، در محاسبات لحاظ نشد. سخت‌افزار پس از تشخیص یک قله‌ی ‌\lr{R} آن را بلافاصله برای سرور می‌فرستد. این مدت‌زمان، زمان ارسال به سرور ($t_{send}$) نامیده شده و به طور متوسط برابر با ۰/۶ ثانیه محاسبه شده است.

قله‌ی \lr{R} تشخیص‌داده‌شده به محض رسیدن به سرور، در پایگاه‌داده ذخیره خواهد شد. مدل زمان انجام این عمل که $t_{store}$ نام دارد در  بدترین حالت ۳۰ میلی‌ثانیه به دست آمد. برای دیدن نتیجه‌ی هر ضربان، لازم است صفحه دوباره بارگذاری شده و عملیات پیش‌بینی انجام شود. صفحه با نرخ یک بار در ثانیه بارگذاری می‌شود و در بدترین حالت، یک ثانیه بعد درخواستی برای پیش‌بینی برچسب ضربانی که هم اکنون در سرور ذخیره شده‌است، از طرف مرورگر به سرور داده خواهد شد. این زمان را $t_{refresh}$ می‌نامیم.

پس از ارسال درخواست، سرور مدتی را صرف پردازش داده‌ی جدید و نمایش نتیجه در صفحه‌ی وب می‌کند. این مدت زمان که $t_{predict}$ نامیده شد نیز در بیشترین حالت ۵۰ میلی‌ثانیه به دست آمد. به این ترتیب می‌توان کل مدت‌زمان پاسخ را طبق معادله‌ی \ref{eq:responseTime} محاسبه کرد.

\begin{equation}
\begin{split}
	& t_{response} = t_{send} + t_{store} + t_{refresh} + t_{predict} \\
	& = 600 ms + 30 ms + 1 s + 50 ms = 1/680 s
\end{split}
\label{eq:responseTime}
\end{equation}
به این ترتیب مدت‌زمانی کم‌تر از ۲ ثانیه برای زمان پاسخ ضمانت می‌شود.
 
\section{معیارهای کارایی نهایی الگوریتم دسته‌بندی}
همان طور که در بخش \ref{subsec:perf} اشاره شد، مهم‌ترین معیاری که در این کار برای سنجش میزان موفقیت الگوریتم دسته‌بندی مورد استفاده قرار دادیم، اندیس \lr{j$\kappa$ } است. روش اعتبارسنجی متقابل مقدار  ۰/۰۰۱ را به عنوان بهترین مقدار برای پارامتر \lr{C} تعیین نمود. پس از قراردادن این مقدار برای \lr{C} و ساخت و ارزیابی مدل، مقدار اندیس \lr{j$\kappa$ } برابر با ۰/۴۲۸ به دست آمد. در جدول ۱-۴ مقادیر به‌دست‌آمده برای دیگر معیارهای کارایی مشاهده می‌شوند.

هم‌چنین مقدار $\kappa$ برابر با ۰/۳۵۶۷ و مقدار اندیس \lr{j } برابر با ۱/۹۹۹۸ به دست آمد. همان طور که اشاره شد، اندیس  \lr{j$\kappa$} به کمک معادله‌ی \ref{eq:jkindex} با استفاده از این مقادیر محاسبه می‌شود.


\begin{table}
\begin{center}
\begin{latin} 
\begin{tabular}{l*{6}{c}r}
Beat Class              & Sensitivity & Precision & Accuracy  \\
\hline
N & 0.7657 & 0.9865 & 0.7831  \\
SVEB            & 0.4717 & 0.2653 & 0.9243  \\
VEB           & 0.7854 & 0.4773 & 0.9299  \\
F    & 0.9201 & 0.0572 & 0.8809  \\
Mean    & 0.7357 & 0.4466 & 0.8796  \\
\end{tabular}
\end{latin}
\caption{نتایج دسته‌بندی در کلاس‌های مختلف ضربان و به صورت میانگین}
\end{center}
\label{table:results}
\end{table}






 
\cchapter{نتیجه‌گیری و کارهای آینده}
\label{chap:conclusion}
\pagebreak
در این کار، یک سیستم تشخیص و دسته‌بندی آریتمی قلبی پیاده‌سازی شد. این سیستم توانایی کارکردن به صورت بی‌درنگ را دارد و در طول مدت‌زمان کم‌تر از ۲ ثانیه، کلاس آریتمی ضربان دریافت‌شده را تشخیص می‌دهد. دقت الگوریتم دسته‌بندی پیاده‌سازی‌شده به طور میانگین ۰/۴۴ به دست آمد، هم‌چنین حساسیت میانگین این دسته‌بندی ۰/۷۳ و  \lr{j$\kappa$ index} آن ۰/۴۳ محاسبه شد. 

این سیستم به دو بخش کلی پیش‌پردازش بر روی سخت‌افزار و پردازش اصلی بر روی سرور تقسیم می‌شود. بخش اول بر روی یک ماژول \lr{ESP8266} پیاده شد که باید به یک سنسور دیجیتال ضربان قلب متصل شود. در این بخش یک الگوریتم تشخیص \lr{QRS} با روش پن-تامپکینز و با هدف تشخیص قله‌های \lr{R} هر ضربان بر روی سیگنال نوار قلب اجرا شد و نتیجه‌ی این پیش‌پردازش به سرور ارسال گردید.

در بخش دوم، یک مدل دسته‌بندی \lr{SVM} به صورت آفلاین آموزش داده شد و سپس مدل آموزش‌داده‌شده به صورت فایل ذخیره شده و به سرور انتقال داده شد. یک کد سمت سرور برای دریافت درخواست‌ها و پیش‌بینی کلاس ضربان‌‌های جدید پیاده‌سازی شد.

الگوریتم دسته‌بندی بر روی پایگاه‌داده‌ی \lr{MIT-BIH} اجرا شد و برای تقسیم داده‌های این پایگاه‌داده به دو مجموعه‌ی آموزش و تست، از الگوی بین‌بیماری بهره گرفته شد. برای ارزیابی کارایی این دسته‌بندی، از معیار اندیس  \lr{j$\kappa$} استفاده شد که نمای مناسبی از میزان کارایی یک الگوریتم دسته‌بندی آریتمی ارایه می‌کند.

از مهم‌ترین نیازمندی‌های این پروژه، سرعت بالا و بی‌درنگ بودن تشخیص آریتمی است که با توجه به زمان پاسخ به‌دست‌آمده، می‌توان گفت این نیازمندی برآورده شده‌است. هم‌چنین قابل‌حمل‌بودن سخت‌افزار همراه بیمار  از نیازمندی‌های دیگر بود که با پیاده‌سازی بخش سخت‌افزاری کار بر روی ماژول \lr{ESP8266} که حجم و مساحت کوچکی دارد، تا حد خوبی برآورده شده‌است.

نیازمندی دیگر این کار، دقت بالای الگوریتم دسته‌بندی ذکر شد. در این کار در مرحله‌ی استخراج ویژگی، تنها از بازه‌های \lr{RR} ضربان‌های قلب استفاده شد و ویژگی‌هایی با استفاده از این بازه‌ها استخراج شدند. دلیل این امر، کارایی بالایی بود که این ویژگی‌ در کارهای گذشته از خود نشان داده‌است. حجم پایین ویژگی‌های استخراج شده و سبک‌تر بودن محاسبات مورد نیاز بر روی سخت‌افزار همراه بیمار  نیز باعث شد ویژگی \lr{RR} مناسب تشخیص داده شود، چرا که ما را به هدف بی‌درنگ‌بودن سیستم و قابل‌حمل‌بودن سخت‌افزار آن نزدیک می‌نماید.

اندیس  \lr{j$\kappa$} به دست آمده در این پروژه، ۰/۴۲۸ است که به نسبت بالاترین اندیس  \lr{j$\kappa$} های به دست آمده در کارهای گذشته (۰/۶۶۳ در \cite{Mar2011}) مقداری پایین‌تر است، اما به نسبت بالاترین مقداری که تنها با استفاده از ویژگی \lr{RR} حاصل شده‌است (۰/۴۳۹ در \cite{Mondejar}) مقداری مناسب است. پس می‌توان گفت به طور کلی به نیازمندی دقت و حساسیت بالای سیستم پاسخ کامل داده نشده‌است، و دلیل این امر، استفاده از ویژگی \lr{RR} به تنهایی بوده‌است.


در کارهای آینده می‌توان از چند مدل \lr{SVM} که هر یک با یک مجموعه از ویژگی‌ها آموزش داده شده اند و ترکیب نتایج آن‌ها، برای ساخت یک مدل \lr{SVM} قوی‌تر بهره برد. در کار پیش رو، در مرحله‌ی تشخیص \lr{QRS}، تنها از ولتاژ یک لید (\lr{MLII}) استفاده شد که واضح‌ترین ترکیب \lr{QRS} را بین لیدهای قلبی نمایش می‌دهد. در آینده می‌توان الگوریتم‌هایی طراحی نمود که از  لیدهای بیشتری بهره می‌برند، و تاثیر این کار را بر بالارفتن دقت تشخیص \lr{QRS} بررسی کرد.


\newpage

%\begin{latin}
\pagestyle{plain}
{
\onehalfspacing
\bibliographystyle{ieeetr-fa}%{chicago-fa}%{plainnat-fa}%
\bibliography{Thesis}
}
%\end{latin}

\newpage

%include{appendix}

\begin{latin}
%\latinfirstPage
\end{latin}
\end{document} 
