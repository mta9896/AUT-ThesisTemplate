
\cchapter{نتایج به‌دست‌آمده }
\label{chap:result}
\pagebreak

در این کار، آزمایش‌هایی با هدف بهینه کردن زمان پاسخ کل الگوریتم و دقت الگوریتم دسته‌بندی انجام شد. هم‌چنین برای ارتباط بهتر کاربران (بیمار و پزشک) با سیستم، یک اپلیکیشن تحت وب برای دسترسی کاربر به نتایج دسته‌بندی طراحی و پیاده‌سازی شد. در ادامه به نتایج به‌دست‌آمده در این بخش‌ها می‌پردازیم.

\section{زمان پاسخ سیستم}
در این کاربرد، زمان پاسخ را مدت‌زمان بین تولید یک ضربان قلب در سیگنال نوار قلب، تا لحظه‌ای که کلاس آن ضربان به کاربر نشان داده می‌شود در نظر گرفته‌ایم. این زمان معیاری برای بررسی سرعت و میزان کارآمدبودن سیستم، به عنوان یک سیستم بی‌درنگ بوده و از این جهت اهمیت بالایی دارد. 

همان‌طور که پیش‌تر توضیح داده شد،در ابتدا سخت‌افزار با دریافت یک ضربان قلب، الگوریتم تشخیص \lr{QRS} را بر روی آن اجرا کرده و قله‌ی \lr{R} را در ضربان تشخیص می‌دهد. مدت‌زمانی که طول می‌کشد تا این عمل انجام شود را $t_{QRS}$ نامیده‌ایم. این زمان به طور متوسط ۳۲۰ میکروثانیه محاسبه شد که به دلیل ناچیزبودن در برابر زمان‌های محاسبه‌شده‌ی دیگر، در محاسبات لحاظ نشد. سخت‌افزار پس از تشخیص یک قله‌ی ‌\lr{R} آن را بلافاصله برای سرور می‌فرستد. این مدت‌زمان، زمان ارسال به سرور ($t_{send}$) نامیده شده و به طور متوسط برابر با ۰/۶ ثانیه محاسبه شده است.

قله‌ی \lr{R} تشخیص‌داده‌شده به محض رسیدن به سرور، در پایگاه‌داده ذخیره خواهد شد. مدل زمان انجام این عمل که $t_{store}$ نام دارد در  بدترین حالت ۳۰ میلی‌ثانیه به دست آمد. برای دیدن نتیجه‌ی هر ضربان، لازم است صفحه دوباره بارگذاری شده و عملیات پیش‌بینی انجام شود. صفحه با نرخ یک بار در ثانیه بارگذاری می‌شود و در بدترین حالت، یک ثانیه بعد درخواستی برای پیش‌بینی برچسب ضربانی که هم اکنون در سرور ذخیره شده‌است، از ظرف مرورگر به سرور داده خواهد شد. این زمان را $t_{refresh}$ می‌نامیم.

پس از ارسال درخواست، سرور مدتی را صرف پردازش داده‌ی جدید و نمایش نتیجه در صفحه‌ی وب می‌کند. این مدت زمان که $t_{predict}$ نامیده شد نیز در بیشترین حالت ۵۰ میلی‌ثانیه به دست آمد. به این ترتیب می‌توان کل مدت‌زمان پاسخ را طبق معادله‌ی \ref{eq:responseTime} محاسبه کرد.

\begin{equation}
\begin{split}
	& t_{response} = t_{send} + t_{store} + t_{refresh} + t_{predict} \\
	& = 600 ms + 30 ms + 1 s + 50 ms = 1/680 s
\end{split}
\label{eq:responseTime}
\end{equation}
به این ترتیب زمان پاسخ کمتر از ۲ ثانیه ضمانت می‌شود که برای کاربرد بی‌درنگ ما مطلوب است.
 
\section{معیارهای کارایی نهایی الگوریتم دسته‌بندی}
همان طور که در بخش \ref{subsec:perf} اشاره شد، مهم‌ترین معیاری که در این کار برای سنجش میزان موفقیت الگوریتم دسته‌بندی مورد استفاده قرار دادیم،  \lr{j$\kappa$ index}  است. روش اعتبارسنجی متقابل مقدار  ۰/۰۰۱ را به عنوان بهترین مقدار برای پارامتر \lr{C} تعیین نمود. پس از قراردادن این مقدار برای \lr{C} و ساخت و ارزیابی مدل، مقدار \lr{j$\kappa$ index}  برابر با ۰/۴۲۸ به دست آمد. در جدول ۱-۴ مقادیر به‌دست‌آمده برای دیگر معیارهای کارایی مشاهده می‌شوند.

هم‌چنین مقدار $\kappa$ برابر با ۰/۳۵۶۷ و مقدار \lr{j index} برابر با ۱/۹۹۹۸ به دست آمد. همان طور که اشاره شد،  \lr{j$\kappa$ index} به کمک معادله‌ی \ref{eq:jkindex} با استفاده از این مقادیر محاسبه می‌شود.


\begin{table}
\begin{center}
\begin{latin} 
\begin{tabular}{l*{6}{c}r}
Beat Class              & Sensitivity & Precision & Accuracy  \\
\hline
N & 0.7657 & 0.9865 & 0.7831  \\
SVEB            & 0.4717 & 0.2653 & 0.9243  \\
VEB           & 0.7854 & 0.4773 & 0.9299  \\
F    & 0.9201 & 0.0572 & 0.8809  \\
Mean    & 0.7357 & 0.4466 & 0.8796  \\
\end{tabular}
\end{latin}
\caption{نتایج دسته‌بندی در کلاس‌های مختلف ضربان و به صورت میانگین}
\end{center}
\label{table:results}
\end{table}





